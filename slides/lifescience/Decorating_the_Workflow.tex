%%%%%%%%%%%%%%%%%%%%%%%%%%%%%%%%%%%%%%%%%%%%%%%%%%%%%%%%%%%%%%%%%%%%%%%%%%%%%%%%
\section{Decorating Workflows - Parameterization}

%%%%%%%%%%%%%%%%%%%%%%%%%%%%%%%%%%%%%%%%%%%%%%%%%%%%%%%%%%%%%%%%%%%%%%%%%%%%%%%%
\begin{frame}
    \frametitle{Outline}
    \begin{columns}[t]
        \begin{column}{.5\textwidth}
            \tableofcontents[sections={1-9},currentsection]
        \end{column}
        \begin{column}{.5\textwidth}
            \tableofcontents[sections={10-18},currentsection]
        \end{column}
    \end{columns}
\end{frame}

%%%%%%%%%%%%%%%%%%%%%%%%%%%%%%%%%%%%%%%%%%%%%%%%%%%%%%%%%%%%%%%%%%%%%%%%%%%%%%%%
\begin{frame}
  \frametitle{What is this about?}
   \question[Questions]{\begin{itemize}
                         \item How can a workflow be made really usable?
                         \item How can I avoid changing the workflow source for every new input?
                        \end{itemize}
                       }
   \docs[Objectives]{\begin{enumerate}
                      \item Learn how to make use of configuration files.  
                      \item Learn how to add parameters to your workflow.
                     \end{enumerate}}
\end{frame}

%%%%%%%%%%%%%%%%%%%%%%%%%%%%%%%%%%%%%%%%%%%%%%%%%%%%%%%%%%%%%%%%%%%%%%%%%%%%%%%%
\subsection{Configuration Files}


%%%%%%%%%%%%%%%%%%%%%%%%%%%%%%%%%%%%%%%%%%%%%%%%%%%%%%%%%%%%%%%%%%%%%%%%%%%%%%%%
\subsection{Command Line vs. Configuration File}

%%%%%%%%%%%%%%%%%%%%%%%%%%%%%%%%%%%%%%%%%%%%%%%%%%%%%%%%%%%%%%%%%%%%%%%%%%%%%%%%
\begin{frame}
  \docs{\texttt{Snakemake} has an extensive command line interface (CLI). \emph{Everything} can be configured on the command line. In addition (almost) everything can be specified in a configuration file.}
  \pause
  \begin{exampleblock}{Which parameter goes where? Some rules of thumb:}
    \begin{columns}[t]
      \begin{column}{0.5\textwidth}
        The CLI:
        \begin{itemize}
         \item frequently changing parameters
         \item short parameters
         \item default parameters
        \end{itemize}
      \end{column}
      \begin{column}{0.5\textwidth}
        The Config File:
        \begin{itemize}
         \item non-volantile parameters specific to your analysis (those which merit mentioning in paper should always go into a file)
         \item long parameters
         \item otherwise workflow specific parameters
        \end{itemize}
      \end{column}
    \end{columns}
  \end{exampleblock}
\end{frame}

%%%%%%%%%%%%%%%%%%%%%%%%%%%%%%%%%%%%%%%%%%%%%%%%%%%%%%%%%%%%%%%%%%%%%%%%%%%%%%%%
\subsection{The Configuration File}

%%%%%%%%%%%%%%%%%%%%%%%%%%%%%%%%%%%%%%%%%%%%%%%%%%%%%%%%%%%%%%%%%%%%%%%%%%%%%%%%
\begin{frame}[fragile]
  \frametitle{The \texttt{Snakemake} \texttt{resources} Section}
  \texttt{Snakemake} rules provide an additional \altverb{resource} section:
  \begin{lstlisting}[language=Python,style=Python]
rule <name>:
   ...
   resources:
      partition='parallel',
      mem_mb=1800,
      cpus_per_task=4
  \end{lstlisting}
  \hint{Note the \textbf{,}!}
  \pause
  \docs{You \emph{may} define \emph{any} resource keyword within any rule.}
\end{frame}

%%%%%%%%%%%%%%%%%%%%%%%%%%%%%%%%%%%%%%%%%%%%%%%%%%%%%%%%%%%%%%%%%%%%%%%%%%%%%%%%
\begin{frame}
  \frametitle{The \texttt{Snakemake} \texttt{resources} Section - its Downside}
  \begin{alertblock}{Every Resource Spec needs a Change per Rule???}
   You might have noticed, this specification per rule is most untidy. \texttt{Snakemake}'s design principle is: ship workflows which run \emph{everywhere} \& \emph{every time}.
   \newline \pause
   Relax: Every bit we can specify:
   \begin{itemize}
    \item in a \texttt{Snakefile},
    \item on the command line,
    \item and re-usable in configuration files!
   \end{itemize}

  \end{alertblock}

\end{frame} 

%%%%%%%%%%%%%%%%%%%%%%%%%%%%%%%%%%%%%%%%%%%%%%%%%%%%%%%%%%%%%%%%%%%%%%%%%%%%%%%%
\begin{frame}
  \frametitle{The Configuration File}
  If observed closely you have seen this file:\newline
            {\tiny \DTsetlength{0.2em}{1em}{0.2em}{0.4pt}{.6pt}
\texttt{\$ tree}
\dirtree{%
.1 /.
.2 config.
.3 samples.yaml .
.2 {other stuff}.
}}
 \pause
 \docs{You can store a yaml file with \emph{your} workflow configuration -- which may be combined with the desinger's configuration.}
 \pause
 \warning{It is better to specify fully qualified paths to your data! The tilde is only there to make a point!}
\end{frame}

%%%%%%%%%%%%%%%%%%%%%%%%%%%%%%%%%%%%%%%%%%%%%%%%%%%%%%%%%%%%%%%%%%%%%%%%%%%%%%%%
\begin{frame}[fragile]
  \frametitle{Using the Configuration File}
  To point to the configuration file, you can add a flag, e.\,g.:
  \begin{lstlisting}[language=Bash, style=Shell]
$ snakemake ... --configfile ./config/config.yaml  
  \end{lstlisting}
  Or, alternatively in your \texttt{Snakefile}:
  \begin{lstlisting}[language=Python,style=Python]
configfile: "config.yaml"
  \end{lstlisting}  
  \question{How do pick up resource parameters in out \texttt{Snakefile}?!}
\end{frame} 

%%%%%%%%%%%%%%%%%%%%%%%%%%%%%%%%%%%%%%%%%%%%%%%%%%%%%%%%%%%%%%%%%%%%%%%%%%%%%%%%
\begin{frame}[fragile]
  \frametitle{Working with Configuration Data}
  \hint{Remember: \texttt{Snakefile}s are Python files.}
  \pause
  Given a file \altverb{config.yaml} with contents:
  \begin{lstlisting}[language=Python,style=Python]
samples:
    A: data/samples/A.fastq
    B: data/samples/B.fastq
  \end{lstlisting}
  we can read our samples within our workflow using
  \begin{lstlisting}[language=Python,style=Python]
configfile: "config.yaml"

rule bcftools_call:
    input:
        bam=expand("sorted_reads/{sample}.bam", 
            sample=@config["samples"]@),
        ...
  \end{lstlisting}
\end{frame}

%%%%%%%%%%%%%%%%%%%%%%%%%%%%%%%%%%%%%%%%%%%%%%%%%%%%%%%%%%%%%%%%%%%%%%%%%%%%%%%%
\begin{frame}[fragile]
  \frametitle{A Comparison}
  \begin{lstlisting}[language=Python,style=Python,basicstyle=\footnotesize]
rule bcftools_call:
    input:
        fa="data/genome.fa",
        bam=expand("sorted_reads/{sample}.bam",
                   sample=config["samples"]),
        bai=expand("sorted_reads/{sample}.bam.bai",
                   sample=config["samples"])
  \end{lstlisting}
  To
  \begin{lstlisting}[language=Python,style=Python,basicstyle=\footnotesize]
rule bcftools_call:
    input:
        fa="data/genome.fa",
        bam=expand("sorted_reads/{sample}.bam",
                    sample=SAMPLES),
        bai=expand("sorted_reads/{sample}.bam.bai",
                   sample=SAMPLES)
  \end{lstlisting}
  \hint{Benefits:
        \begin{itemize}
         \item no more searching in and tinkering with source code
         \item clean overview in configuration files, which are there where you expect them to be.
        \end{itemize}}
\end{frame}

%%%%%%%%%%%%%%%%%%%%%%%%%%%%%%%%%%%%%%%%%%%%%%%%%%%%%%%%%%%%%%%%%%%%%%%%%%%%%%%%
\begin{frame}[fragile]
  \frametitle{Input Functions}
  Since we have stored the path to the FASTQ files in the config file, we can also generalize the rule \altverb{bwa_map} to use these paths. This case is different to the rule \altverb{bcftools_call} we modified above. To understand this, it is important to know that \texttt{Snakemake} workflows are executed in three phases.
  \begin{itemize}[<+->]
   \item In the \emph{initialization phase}, the files defining the workflow are parsed and all rules are instantiated.
   \item In the \emph{DAG phase}, the directed acyclic dependency graph of all jobs is built by filling wildcards and matching input files to output files.
   \item In the \emph{scheduling phase}, the DAG of jobs is executed, with jobs started according to the available resources.
  \end{itemize}
\end{frame}

%%%%%%%%%%%%%%%%%%%%%%%%%%%%%%%%%%%%%%%%%%%%%%%%%%%%%%%%%%%%%%%%%%%%%%%%%%%%%%%%
\begin{frame}[fragile]
  \frametitle{Input Functions - II}
    \begin{alertblock}{Why we cannot do the same all over the Workflow}
    We cannot determine the FASTQ paths for rule \altverb{bwa_map} from the config file in this phase, \emph{because we don’t even know which jobs will be generated from that rule}. Instead, we need to defer the determination of input files to the DAG phase. This can be achieved by specifying an input function instead of a string as inside of the input directive.
  \end{alertblock}
  \begin{lstlisting}[language=Python,style=Python]
def get_bwa_map_input_fastqs(wildcards):
    return config["samples"][wildcards.sample]

rule bwa_map:
    input:
        "data/genome.fa",
        get_bwa_map_input_fastqs
  \end{lstlisting}
\end{frame}

%%%%%%%%%%%%%%%%%%%%%%%%%%%%%%%%%%%%%%%%%%%%%%%%%%%%%%%%%%%%%%%%%%%%%%%%%%%%%%%%
\begin{frame}[fragile]
  \frametitle{\HandsOn{Adding a new Sample}}
  In the \altverb{data/samples} folder, there is an additional sample \altverb{C.fastq}. Add that sample to the config file and see how \texttt{Snakemake} wants to recompute the part of the workflow belonging to the new sample, when invoking with 
  \begin{lstlisting}[language=Bash, style=Shell]
$ snakemake -n --forcerun bcftools_call
  \end{lstlisting}
  Copy the file XXX as your \texttt{Snakefile}. 
\end{frame}

%%%%%%%%%%%%%%%%%%%%%%%%%%%%%%%%%%%%%%%%%%%%%%%%%%%%%%%%%%%%%%%%%%%%%%%%%%%%%%%%
\begin{frame}[fragile]
  \frametitle{Rule Parameters}
  \warning{Frequently, party programs cannot be used with their defaults, additional parameters are required.}
   In our workflow, it is reasonable to annotate aligned reads with so-called read groups, that contain metadata like the sample name. \newline
   To do so, we add to our \altverb{bwa_map}-rule:
   \begin{lstlisting}[style=Plain,basicstyle=\footnotesize]
rule bwa_map:
    input:
        "data/genome.fa",
        get_bwa_map_input_fastqs
    output:
        "mapped_reads/{sample}.bam"
    params:
        rg=r"@RG\tID:{sample}\tSM:{sample}"
    shell:
        "bwa mem -R '{params.rg}' {input} "
        "| samtools view -Sb - > {output}"
   \end{lstlisting}
\end{frame}

%%%%%%%%%%%%%%%%%%%%%%%%%%%%%%%%%%%%%%%%%%%%%%%%%%%%%%%%%%%%%%%%%%%%%%%%%%%%%%%%
\begin{frame}[fragile]
  \frametitle{Rule Parameters}
  This is usually part of the configuration file(s), too.\newline
  In your config file:
  \begin{lstlisting}[language=Python,style=Python]
bwa_map:
   rg=xxx
  \end{lstlisting}
  and in your code:
  \begin{lstlisting}[language=Python,style=Python]
rule bwa_map:
  params:
     rg=config['bwa_map']['rg']
  \end{lstlisting}
  Now, your workflow is configurable!
\end{frame}

%TODO: enter a slide with a figure illustrating the command line to easy the flow
%%%%%%%%%%%%%%%%%%%%%%%%%%%%%%%%%%%%%%%%%%%%%%%%%%%%%%%%%%%%%%%%%%%%%%%%%%%%%%%%
\subsection{The Command Line}

%%%%%%%%%%%%%%%%%%%%%%%%%%%%%%%%%%%%%%%%%%%%%%%%%%%%%%%%%%%%%%%%%%%%%%%%%%%%%%%%
\begin{frame}[fragile]
  \frametitle{Executor Selection}
  \texttt{Snakemake} lets you select various executors. Not happy with \mogon? Take another cluster or \lhref{https://snakemake.readthedocs.io/en/stable/executor_tutorial/tutorial.html}{Google Lifescience, Tibanna, Kubernetes, \ldots} \newline
  We may happily select the most prominent HPC batch system, the one running on \mogon, too:
  \begin{lstlisting}[language=Bash, style=Shell]
$ snakemake --slurm
  \end{lstlisting}
  Now, \emph{every} rule will submit its jobs as HPC compute jobs.
  \hint{We will learn how to avoid this, soon-ish.}
\end{frame}

%%%%%%%%%%%%%%%%%%%%%%%%%%%%%%%%%%%%%%%%%%%%%%%%%%%%%%%%%%%%%%%%%%%%%%%%%%%%%%%%
\begin{frame}[fragile]
  \frametitle{Default Resources for \texttt{SLURM}}
  Without specifying our SLURM-account and a (default) partition, submitting batch jobs will fail. \texttt{Snakemake} allows to define so-called default resources (using \altverb{--default-resources}). With them our minimal command line becomes:
  \begin{lstlisting}[language=Bash, style=Shell, breaklines=true]
$ snakemake --slurm \
> --default-resources slurm_account=m2_jgu-ngstraining \
>                     slurm_partition=smp
  \end{lstlisting}
  \hint{Please notice the missing quotation marks! All arguments belong to one parameter.}
  \docs{We wrote \texttt{slurm\_...} to distinguish from other non-SLURM accounts and similar stuff.}
\end{frame}

%%%%%%%%%%%%%%%%%%%%%%%%%%%%%%%%%%%%%%%%%%%%%%%%%%%%%%%%%%%%%%%%%%%%%%%%%%%%%%%%
\begin{frame}[fragile]
  \frametitle{The Beauty of Clusters}
  A HPC cluster:
  \begin{itemize}
   \item offers great resources
   \item may hold jobs pending until resources are available!
  \end{itemize}
  \pause
  \texttt{Snakemake} of a semaphore to throttle resource usage, called \altverb{--jobs/-j}. We can now write \altverb{-j unlimited} in place for \altverb{--cores 1}. Let us try
  \begin{lstlisting}[language=Bash, style=Shell, basicstyle=\footnotesize]
$ snakemake --slurm \
> --default-resources \ 
>   slurm_account=m2_jgu-ngstraining \
>   slurm_partition=smp
> -j unlimited
  \end{lstlisting}
  together.
  \hint{You might want to invoke the \texttt{clean}-rule, first.}
\end{frame}

%%%%%%%%%%%%%%%%%%%%%%%%%%%%%%%%%%%%%%%%%%%%%%%%%%%%%%%%%%%%%%%%%%%%%%%%%%%%%%%%
\begin{frame}[fragile]
  \frametitle{\texttt{SLURM} supporting Features of \texttt{Snakemake}}
  \texttt{Snakemake} will
  \begin{itemize}[<+->]
   \item hold track of your job status (frequency of checks can be adjusted)
   \item cancel your jobs, when itself is stopped
   \item track resource consumption (generated with \altverb{--report [FILE]})
   \item with \altverb{-j unlimited} we allow for an unlimited jumber of spawned jobs!
  \end{itemize}
  \pause
  \warning{Unlimited number of jobs may yield in I/O contention and too many calls to check the job status. Use with care for both issues there is a remedy, which we will meet later!}
\end{frame}
