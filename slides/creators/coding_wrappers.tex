

\section{Coding own Wrappers}

\begin{frame}{Creating new Wrappers}
    Creating snakemake wrappers is as easy as creating the three constituent parts:
    \begin{description}[<+->]
        \item[\texttt{meta.yaml}:] A documentation file, containing information about the tool and the wrapper
        \item[\texttt{environment.yaml}:] A conda environment to install the tool and its dependencies
        \item[\texttt{wrapper.py}:] The python code to run the tool 
    \end{description}
    \only<1>{
        Contents of the documentation:
        \begin{itemize}
            \item The name of the wrapper and the tool it wraps
            \item Links to the tool source code
            \item Description of the input and output files that should be supplied by any rule using the wrapper
            \item Optionally, some documentation of custom arguments like \altverb{extra}
            \item When contributing to the community: Authors of the wrapper
        \end{itemize}
    }
    \only<2>{
        Software Environment:
        \begin{itemize}
            \item Install all necessary tools and dependencies to run the wrapper code (including python libraries)
            \item Preferably tool versions should be fixed, s.t. later usage of the same wrapper results in the same output
        \end{itemize}
    }
    \only<3>{
        The wrapper code:
        \begin{itemize}
            \item Translate the snakemake rule parameters into a valid invocation of the tool
            \item Apply necessary pre- and postprocessing steps to guarantee valid output
        \end{itemize}
    }
    \only<4>{
        \begin{docs}
            Wrappers can be improved by providing \altverb{Unit Tests} that can validate the wrapper code
            and tool versions with known output.
        \end{docs}
    }
\end{frame}

\begin{frame}[fragile]{Creating New Wrappers: Python Code}
    \altverb{wrapper.py} is the core component of any wrapper as it takes care of running the actual
    program.
    \begin{onlyenv}<2>
        \begin{lstlisting}[language=Python,style=Python,gobble=12]
            # Core setup of a `wrapper.py` script
            # 1. Python Imports
            import tempfile
            import os
            from snakemake.shell import shell
        \end{lstlisting}
    \end{onlyenv}
    \begin{onlyenv}<3>
        \begin{lstlisting}[language=Python,style=Python,gobble=12]
            # Core setup of a `wrapper.py` script
            # 2. Fetch the snakemake input and output parameter
            extra = snakemake.params.get("extra", "")
        \end{lstlisting}
        The snakemake object has the attributes \altverb{input}, \altverb{output} and \altverb{params}
        containing the respective rule parameter.
    \end{onlyenv}
    \begin{onlyenv}<4>
        \begin{lstlisting}[language=Python,style=Python,gobble=12]
            # Core setup of a `wrapper.py` script
                                            Default value
                                                 @vvvv@
            extra = snakemake.params.get("extra", "")
                                         @^^^^^^^@
                                    Name of the parameter
        \end{lstlisting}
        The snakemake object has the attributes \altverb{input}, \altverb{output} and \altverb{params}
        containing the respective rule parameter.
    \end{onlyenv}
    \begin{onlyenv}<5>
        \begin{lstlisting}[language=Python,style=Python,gobble=12]
            # Core setup of a `wrapper.py` script
            extra = snakemake.params.get("extra", "")
            reads = snakemake.input.get("reads", [])
            output = snakemake.output[0]
        \end{lstlisting}
        \begin{itemize}
            \item These values can also be addressed as arrays
            \item The wrapper must take care of checking for errors in the input and raise Exceptions if necessary
        \end{itemize}
    \end{onlyenv}
    \begin{hint}
        The \altverb{snakemake} object provides access to snakemake variables from inside the
        wrapper script.
    \end{hint}
\end{frame}

\begin{frame}[fragile]{Creating New Wrappers: Helper Functions}
    Snakemake provides several helper functions that make it easier to code wrappers.
    \begin{lstlisting}[language=Python,style=Python,gobble=8]
        # Snakemake Helper Functions

        # Run a specific tool on the command line.
        snakemake.shell("/bin/true")

        # Format the log statement of the current rule, s.t. it can be applied to a shell command.
        snakemake.log_fmt_shell(stdout=True, stderr=True, append=False)

        # Define the executable used to launch your shell commands.
        snakemake.shell.executable("/bin/bash")
    \end{lstlisting}
\end{frame}

\begin{frame}{Assembling the new Wrapper}
    With these ingredients, you can write wrappers for your own tools.
    Recap of the steps required to create a new wrapper:
    \begin{itemize}
        \item Pick the tool and write the wrapper
        \item Test the wrapper and document how to use it
        \item Optionally, write some unit testing code for the wrapper
        \item Use it in your workflows
    \end{itemize}
    \begin{docs}
        If you write new wrappers, please contribute them to the community effort.
    \end{docs}
\end{frame}

\begin{frame}{Contributing Wrappers to the Community}
    New wrappers are always welcome at \lhref{https://github.com/snakemake/snakemake-wrappers}{snakemake-wrappers}.
    In order to contribute a new wrapper, please provide:
    \begin{itemize}
        \item A Wrapper, including \altverb{meta.yaml}, \altverb{environment.yaml} and \altverb{wrapper.py}
        \item A simple unit test with some example data
    \end{itemize}
    \begin{hint}
        Your example serves as the main documentation for the wrapper, so make sure to document
        the usage of any custom parameters, input and output variables.
    \end{hint}
\end{frame}
