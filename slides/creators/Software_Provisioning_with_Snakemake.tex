%%%%%%%%%%%%%%%%%%%%%%%%%%%%%%%%%%%%%%%%%%%%%%%%%%%%%%%%%%%%%%%%%%%%%%%%%%%%%%%%
\section{Software Provisioning with Snakemake}

%%%%%%%%%%%%%%%%%%%%%%%%%%%%%%%%%%%%%%%%%%%%%%%%%%%%%%%%%%%%%%%%%%%%%%%%%%%%%%%%
\begin{frame}
	\frametitle{Outline}
	\begin{columns}[t]
		\begin{column}{.5\textwidth}
            \tableofcontents[sections={1-7},currentsection]
		\end{column}
		\begin{column}{.5\textwidth}
            \tableofcontents[sections={8-15},currentsection]
		\end{column}
	\end{columns}
\end{frame}

%%%%%%%%%%%%%%%%%%%%%%%%%%%%%%%%%%%%%%%%%%%%%%%%%%%%%%%%%%%%%%%%%%%%%%%%%%%%%%%%
\begin{frame}
	\frametitle{What is this about?}
	\begin{question}[Questions]
		\begin{itemize}
			\item Conda, Container, Environment Modules?! What is the Fuzz?
		\end{itemize}
	\end{question}
	\begin{docs}[Objectives]
		\begin{enumerate}
			\item Learn about the possible provisioning methods of \Snakemake.
			\item Learning to distinguish these methods.
		\end{enumerate}
	\end{docs}
\end{frame}

%%%%%%%%%%%%%%%%%%%%%%%%%%%%%%%%%%%%%%%%%%%%%%%%%%%%%%%%%%%%%%%%%%%%%%%%%%%%%%%%
\subsection{Your Options}

%%%%%%%%%%%%%%%%%%%%%%%%%%%%%%%%%%%%%%%%%%%%%%%%%%%%%%%%%%%%%%%%%%%%%%%%%%%%%%%%
\begin{frame}
  \frametitle{Snakemake's Software Provisioning - Overview}
  \Snakemake{} basically offers 3 software deployment methods:
  \begin{itemize}[<+->]
    \item Container
    \item Environment Modules
    \item Conda (and its derivatives)
  \end{itemize}
\end{frame}

%%%%%%%%%%%%%%%%%%%%%%%%%%%%%%%%%%%%%%%%%%%%%%%%%%%%%%%%%%%%%%%%%%%%%%%%%%%%%%%%
\subsection{AppTainer}

%%%%%%%%%%%%%%%%%%%%%%%%%%%%%%%%%%%%%%%%%%%%%%%%%%%%%%%%%%%%%%%%%%%%%%%%%%%%%%%%
\begin{frame}[fragile]
  \frametitle{Using Containers}
  Within a rule we can use the \altverb{container} directive to select a container image
  \begin{lstlisting}[language=Python,style=Python]
rule NAME:
  input: "table.txt"
  output: "plots/myplot.pdf"
  @container:@
    @"docker://joseespinosa/docker-r-ggplot2"@
  script:
    "scripts/plot-stuff.R"
  \end{lstlisting}
  \begin{docs}{Allowed Container Flavours}
  	Allowed image urls need to be supported by AppTainer (e.\,g., \altverb{shub://} and \altverb{docker://}). \altverb{docker://} is preferred.
  \end{docs}	
\end{frame}

%%%%%%%%%%%%%%%%%%%%%%%%%%%%%%%%%%%%%%%%%%%%%%%%%%%%%%%%%%%%%%%%%%%%%%%%%%%%%%%%
\begin{frame}[fragile]
  \frametitle{Using Containers II}
  When executing \Snakemake{} with
  \begin{lstlisting}[language=Bash, style=Shell]
snakemake --software-deployment-method apptainer
# or the shorthand version
snakemake --sdm apptainer
  \end{lstlisting}
   it will execute the job within a container that is spawned from the given image.
\end{frame}

%%%%%%%%%%%%%%%%%%%%%%%%%%%%%%%%%%%%%%%%%%%%%%%%%%%%%%%%%%%%%%%%%%%%%%%%%%%%%%%%
\begin{frame}[fragile]
  \frametitle{\Interlude{Containerizing your Workflow}}
  A \Snakemake{} workflow with conda environments for each rule can automatically generate a container image specification (as a Dockerfile) that contains all required environments:
  \begin{lstlisting}[language=Bash, style=Shell]
snakemake --containerize > Dockerfile
  \end{lstlisting}
  \begin{question}{Which is the Purpose?}
  	\begin{itemize}
  	  \item archive a container for publication.
  	  \item deploy and ship a container to a $3^{\mathsf{rd}}$ party
  	\end{itemize}
  \end{question}
\end{frame}

%%%%%%%%%%%%%%%%%%%%%%%%%%%%%%%%%%%%%%%%%%%%%%%%%%%%%%%%%%%%%%%%%%%%%%%%%%%%%%%%
\subsection{Conda}

%%%%%%%%%%%%%%%%%%%%%%%%%%%%%%%%%%%%%%%%%%%%%%%%%%%%%%%%%%%%%%%%%%%%%%%%%%%%%%%%
\begin{frame}[fragile]
  \frametitle{Integrated Package Management}
  \begin{columns}
  	\begin{column}{0.5\textwidth}
  	  In rules we can use the \altverb{conda} directive to specify our environment.
  	  \begin{lstlisting}[language=Python,style=Python,basicstyle=\footnotesize]
rule NAME:
  input:
  	"table.txt"
  output:
  	"plots/myplot.pdf"
  @conda:@
  	"envs/ggplot.yaml"
  script:
  	"scripts/plot-stuff.R"
  	  \end{lstlisting}
  	\end{column}
    \begin{column}{0.5\textwidth}
       with the following \lhref{}{environment definition}:
       \begin{lstlisting}[style=Plain]
channels:
- r
dependencies:
- r=3.3.1
- r-ggplot2=2.1.0
      \end{lstlisting}   
    \end{column}
  \end{columns}	
\end{frame}

%%%%%%%%%%%%%%%%%%%%%%%%%%%%%%%%%%%%%%%%%%%%%%%%%%%%%%%%%%%%%%%%%%%%%%%%%%%%%%%%
\begin{frame}[fragile]
  \frametitle{Important Conda Flags}
  To use Conda add
  \begin{lstlisting}[language=Bash, style=Shell]
--software-deployment-method conda 
# or short
--sdm conda
  \end{lstlisting}
  \pause
  It is possible to use different Conda frontends:
  \begin{lstlisting}[language=Bash, style=Shell]
--conda-frontend mamba # or micromamba
  \end{lstlisting}
\end{frame}

%%%%%%%%%%%%%%%%%%%%%%%%%%%%%%%%%%%%%%%%%%%%%%%%%%%%%%%%%%%%%%%%%%%%%%%%%%%%%%%%
\subsection{Environment Modules}

%%%%%%%%%%%%%%%%%%%%%%%%%%%%%%%%%%%%%%%%%%%%%%%%%%%%%%%%%%%%%%%%%%%%%%%%%%%%%%%%
\begin{frame}[fragile]
  \frametitle{Using Environment Modules}
  You can mix Conda and Environment Modules:
  \begin{lstlisting}[language=Python,style=Python]
rule bwa:
  input:
    "genome.fa"
    "reads.fq"
  output:
    "mapped.bam"
  conda:
    "envs/bwa.yaml"
  @envmodules:@
    "bio/bwa/0.7.9",
    "bio/samtools/1.9"
  shell:
    "bwa mem {input} | samtools view -Sbh - > {output}"
  \end{lstlisting}
\end{frame}

%%%%%%%%%%%%%%%%%%%%%%%%%%%%%%%%%%%%%%%%%%%%%%%%%%%%%%%%%%%%%%%%%%%%%%%%%%%%%%%%
\begin{frame}[fragile]
  \frametitle{Using Environment Modules II}
  When \Snakemake{} is executed with 
  \begin{lstlisting}[language=Bash, style=Shell]
snakemake --use-envmodules
  \end{lstlisting}
  it will load the defined modules in the given order, instead of using the also defined conda environment.
  \pause
  \begin{hint}
  	It is recommended to specify Conda environments alongside - only then containerizing a workflow works!
  \end{hint}
\end{frame}

%%%%%%%%%%%%%%%%%%%%%%%%%%%%%%%%%%%%%%%%%%%%%%%%%%%%%%%%%%%%%%%%%%%%%%%%%%%%%%%%
\subsection{Summing Up}

%%%%%%%%%%%%%%%%%%%%%%%%%%%%%%%%%%%%%%%%%%%%%%%%%%%%%%%%%%%%%%%%%%%%%%%%%%%%%%%%
\begin{frame}[fragile]
  \frametitle{Software Provisioning Summary}
  You can
  \begin{itemize}
  	\item manually install environment (done in this course)
  	\item use \Snakemake's features for conda, env. modules, containers
  \end{itemize}
   See 
   \begin{lstlisting}[language=Bash, style=Shell]
snakemake --help
   \end{lstlisting}
   for an overview or \lhref{https://snakemake.readthedocs.io/en/stable/snakefiles/deployment.html}{the official documentation}.
\end{frame} 