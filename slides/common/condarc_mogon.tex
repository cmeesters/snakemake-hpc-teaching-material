%%%%%%%%%%%%%%%%%%%%%%%%%%%%%%%%%%%%%%%%%%%%%%%%%%%%%%%%%%%%%%%%%%%%%%%%%%%%%%% 
\begin{frame}[fragile]
  \frametitle{Reducing Search Overhead - the \texttt{.condarc}-File}
  On many HPC clusters the number of Conda channels (the repositories) is reduced by the whitelisting. Regardless of your cluster settings, it helps to reduce the search time with a resource file, including  a number of definitions, \emph{before} starting:
  \begin{lstlisting}[language=Bash, style=Shell, basicstyle=\tiny]
$ cat .condarc
create_default_packages:
  - setuptools # since Python is often needed
# these channels cover most of the required software
channels:
  - conda-forge
  - bioconda
  - defaults
  - r
proxy_servers: # in case a proxy server is present
  http: http://webproxy.zdv.uni-mainz.de:8888
ssl_verify: false
auto_update_conda: false
always_yes: true # avoid confirmation(s)
env_prompt: '($(basename {default_env})) '
  \end{lstlisting}
  To obtain the same resource file, run:
  \begin{lstlisting}[language=Bash, style=Shell, basicstyle=\footnotesize]
$ cp condarc ~/.condarc
  \end{lstlisting}
  More on \altverb{.condarc} on the \lhref{https://conda.io/projects/conda/en/latest/user-guide/configuration/use-condarc.html}{official Conda documentation site}
\end{frame}
