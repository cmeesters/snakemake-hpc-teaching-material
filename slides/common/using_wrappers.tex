%%%%%%%%%%%%%%%%%%%%%%%%%%%%%%%%%%%%%%%%%%%%%%%%%%%%%%%%%%%%%%%%%%%%%%%%%%%%%%%%
\section{Snakemake Wrappers}

%%%%%%%%%%%%%%%%%%%%%%%%%%%%%%%%%%%%%%%%%%%%%%%%%%%%%%%%%%%%%%%%%%%%%%%%%%%%%%%%
\subsection{Basics}

%%%%%%%%%%%%%%%%%%%%%%%%%%%%%%%%%%%%%%%%%%%%%%%%%%%%%%%%%%%%%%%%%%%%%%%%%%%%%%%%
\begin{frame}
    \frametitle{Outline}
    \begin{columns}[t]
        \begin{column}{.5\textwidth}
            \tableofcontents[sections={1-9},currentsection]
        \end{column}
        \begin{column}{.5\textwidth}
            \tableofcontents[sections={10-18},currentsection]
        \end{column}
    \end{columns}
\end{frame}

%%%%%%%%%%%%%%%%%%%%%%%%%%%%%%%%%%%%%%%%%%%%%%%%%%%%%%%%%%%%%%%%%%%%%%%%%%%%%%%%
\begin{frame}
    \frametitle{What is this about?}
    \begin{question}[Questions]
        \begin{itemize}
            \item What is a snakemake Wrapper?
            \item How can snakemake wrappers be used to write reproducible workflows?
        \end{itemize}
    \end{question}
    \begin{docs}[Objectives]
        \begin{enumerate}
            \item Introduce the concept of snakemake wrappers
            \item Include snakemake wrappers in your workflows
            \item Reproducibility of snakemake wrappers
        \end{enumerate}
    \end{docs}
\end{frame}

%%%%%%%%%%%%%%%%%%%%%%%%%%%%%%%%%%%%%%%%%%%%%%%%%%%%%%%%%%%%%%%%%%%%%%%%%%%%%%%%
\begin{frame}{Introduction to Wrappers}
    For a single tool, Snakemake Wrappers do:
    \begin{itemize}[<+->]
        \item Document usage
        \item Describe the dependencies
        \item Provide a way to call the tool with information from snakemake
    \end{itemize}
    This setup makes snakemake wrappers reusable across multiple workflows.
\end{frame}

%%%%%%%%%%%%%%%%%%%%%%%%%%%%%%%%%%%%%%%%%%%%%%%%%%%%%%%%%%%%%%%%%%%%%%%%%%%%%%%%
\begin{frame}{Introduction to Wrappers}
    Wrappers consist of three components:
    \begin{enumerate}
        \item A description of their usage (\altverb{meta.yaml})
        \item A conda environment description of all dependencies of the tool (\altverb{environment.yaml})
        \item A python script that invokes the tool (\altverb{wrapper.py})
    \end{enumerate}
    Since they are reusable, there is a large repository online of tried and tested 
    wrappers written by the community: \lhref{https://github.com/snakemake/snakemake-wrappers}{snakemake-wrappers}
\end{frame}

%%%%%%%%%%%%%%%%%%%%%%%%%%%%%%%%%%%%%%%%%%%%%%%%%%%%%%%%%%%%%%%%%%%%%%%%%%%%%%%%
\begin{frame}[fragile]{Using Wrappers}
    Using snakemake wrappers can be achieved easily:
    \begin{lstlisting}[language=Python,style=Python]
rule bwa_map:
    input:
        reads=["input/{sample}_1.fq.gz", "input/{sample}_2.fq.gz"],
        idx=multiext("genome", ".amb", ".ann", ".bwt", ".pac", ".sa"),
    output:
        "output/{sample}.bam"
    params:
        extra=r"-R '@RG\tID:{sample}\tSM:{sample}'",
        sorting="none",  # Can be 'none', 'samtools' or 'picard'.
        sort_order="queryname",  # Can be 'queryname' or 'coordinate'.
        sort_extra="",  # Extra args for samtools/picard.
    threads: 8
    wrapper:
        "v3.2.0/bio/bwa/mem"
    \end{lstlisting}
    \begin{docs}
        This example uses the community snakemake-wrappers repository \altverb{bwa-mem} wrapper.
    \end{docs}
\end{frame}

%%%%%%%%%%%%%%%%%%%%%%%%%%%%%%%%%%%%%%%%%%%%%%%%%%%%%%%%%%%%%%%%%%%%%%%%%%%%%%%%
\begin{frame}[fragile]{Using Wrappers: In- and Output}
    Input and Output of the rules has to match the definition of the snakemake wrapper.
    \begin{lstlisting}[language=Python,style=Python]
rule bwa_map:
    input:
        reads=["input/{sample}_1.fq.gz", "input/{sample}_2.fq.gz"],
        idx=multiext("genome", ".amb", ".ann", ".bwt", ".pac", ".sa"),
    output:
        "output/{sample}.bam"
    ...
    \end{lstlisting}
    \begin{docs}
        The \altverb{bwa-mem} wrapper expects an input object with two attributes
        \altverb{reads} and \altverb{idx} and writes to a single output file.
    \end{docs}
\end{frame}

%%%%%%%%%%%%%%%%%%%%%%%%%%%%%%%%%%%%%%%%%%%%%%%%%%%%%%%%%%%%%%%%%%%%%%%%%%%%%%%%
\begin{frame}[fragile]{Using Wrappers: Parameters}
    Parameters may be used by wrappers to influence the behaviour, settings or resource usage.
    \begin{lstlisting}[language=Python,style=Python]
rule bwa_map:
    ...
    params:
        extra=r"-R '@RG\tID:{sample}\tSM:{sample}'",
        sorting="none",  # Can be 'none', 'samtools' or 'picard'.
        sort_order="queryname",  # Can be 'queryname' or 'coordinate'.
        sort_extra="",  # Extra args for samtools/picard.
    threads: 8
    ...
    \end{lstlisting}
    \begin{docs}
        \altverb{bwa-mem} supports custom arguments like \altverb{sort_order} and
        the common \altverb{extra} flag, that is used to supply command line parameters
        for the invocation of the tool. The \altverb{threads} and \altverb{resources}
        settings from a snakemake rule are also commonly used to set memory or parallelization
        parameters for each tool.
    \end{docs}
\end{frame}

%%%%%%%%%%%%%%%%%%%%%%%%%%%%%%%%%%%%%%%%%%%%%%%%%%%%%%%%%%%%%%%%%%%%%%%%%%%%%%%%
\begin{frame}[fragile]{Using Wrappers: Calling a Wrapper}
    The \altverb{wrapper} directive signals that snakemake should execute this rule with
    a wrapper.
    \begin{lstlisting}[language=Python,style=Python]
rule bwa_map:
    ...
    wrapper:
        "v3.2.0/bio/bwa/mem"
    \end{lstlisting}
    \begin{docs}
        The path to the tool is given relative to the repository or path given by the \altverb{--wrapper-prefix}
        flag, which defaults to the \lhref{https://github.com/snakemake/snakemake-wrappers}{snakemake-wrappers} repository.
    \end{docs}
\end{frame}

%%%%%%%%%%%%%%%%%%%%%%%%%%%%%%%%%%%%%%%%%%%%%%%%%%%%%%%%%%%%%%%%%%%%%%%%%%%%%%%%
\begin{frame}{Advantages of Wrappers}
    Using wrappers has many great advantages compared to \altverb{run} or \altverb{shell}:
    \begin{itemize}[<+->]
        \item Code is reusable across workflows and can be shared with others
        \item Complex invocations of tools do not clutter your snakemake workflow definitions
        \item With versioned wrappers, you workflow is easily reproducible
        \item Community wrappers often get you started quickly using a new tool correctly
    \end{itemize}
\end{frame}
