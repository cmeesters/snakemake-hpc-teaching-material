%%%%%%%%%%%%%%%%%%%%%%%%%%%%%%%%%%%%%%%%%%%%%%%%%%%%%%%%%%%%%%%%%%%%%%%%%%%%%%%%
\section{Parametizing your Workflow - II}

%%%%%%%%%%%%%%%%%%%%%%%%%%%%%%%%%%%%%%%%%%%%%%%%%%%%%%%%%%%%%%%%%%%%%%%%%%%%%%%%
\begin{frame}
    \frametitle{Outline}
    \begin{columns}[t]
        \begin{column}{.5\textwidth}
            \tableofcontents[sections={1-9},currentsection]
        \end{column}
        \begin{column}{.5\textwidth}
            \tableofcontents[sections={10-18},currentsection]
        \end{column}
    \end{columns}
\end{frame}

%%%%%%%%%%%%%%%%%%%%%%%%%%%%%%%%%%%%%%%%%%%%%%%%%%%%%%%%%%%%%%%%%%%%%%%%%%%%%%%%
\begin{frame}
  \frametitle{What is this about?}
  \begin{question}[Questions]
   	\begin{itemize}
      \item How do we add execution parameters?
      \item How do we tune scientific parameters?
    \end{itemize}
  \end{question}
   \begin{docs}[Objectives]
   	 \begin{enumerate} 
        \item Learn to use parameters relevant for the batch systems.
        \item Learn how to tune \Snakemake{} on the command line.
        \item Learn how to tune \Snakemake{} with configuration files.
    \end{enumerate}
  \end{docs}
\end{frame}
