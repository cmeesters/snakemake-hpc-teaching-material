%%%%%%%%%%%%%%%%%%%%%%%%%%%%%%%%%%%%%%%%%%%%%%%%%%%%%%%%%%%%%%%%%%%%%%%%%%%%%%%%
\section{Snakefiles as Python-Code}

%%%%%%%%%%%%%%%%%%%%%%%%%%%%%%%%%%%%%%%%%%%%%%%%%%%%%%%%%%%%%%%%%%%%%%%%%%%%%%%%
\begin{frame}
  \frametitle{What is this about?}
   \question[Questions]{\begin{itemize}
                         \item How can I automatically manage dependencies and outputs?
                         \item How can I use Python code to add features to my pipeline?
                        \end{itemize}
                       }
   \docs[Objectives]{\begin{enumerate}
                      \item Use variables, functions, and imports in a Snakefile. 
                      \item Learn to use the \texttt{run} action to execute Python code as an action.
                     \end{enumerate}}
\end{frame}

%%%%%%%%%%%%%%%%%%%%%%%%%%%%%%%%%%%%%%%%%%%%%%%%%%%%%%%%%%%%%%%%%%%%%%%%%%%%%%%%
\subsection{Python Code in Snakefiles}

%%%%%%%%%%%%%%%%%%%%%%%%%%%%%%%%%%%%%%%%%%%%%%%%%%%%%%%%%%%%%%%%%%%%%%%%%%%%%%%%
\begin{frame}[fragile]
  \frametitle{The Issues we still have}
  Our workflow still has repeated content, e.\,g. the names of output files/dependencies. Our \texttt{zipf\_test} rule is extremely clunky. What happens if we want to analyse \texttt{books/sierra.txt} as well? We’d have to update everything!
  \begin{lstlisting}[language=Python,style=Python, basicstyle=\tiny]
rule zipf_test:
    input:  'scripts/zipf_test.py', 'abyss.dat',
            'last.dat', 'isles.dat'
    output: 'results.txt'
    shell:  'python {input[0]} {input[1]} {input[2]}'
            ' {input[3]} > {output}'
  \end{lstlisting}
  You probably noticed: all of our rules are using Python strings! Other data structures work too — let’s try a list:
  \begin{lstlisting}[language=Python,style=Python, basicstyle=\tiny]
rule zipf_test:
    input:
        zipf='scripts/zipf_test.py',
        books=['abyss.dat', 'last.dat', 'isles.dat']
    output: 'results.txt'
    shell:  'python {input.zipf} {input.books} > {output}'
  \end{lstlisting}
\end{frame}

%%%%%%%%%%%%%%%%%%%%%%%%%%%%%%%%%%%%%%%%%%%%%%%%%%%%%%%%%%%%%%%%%%%%%%%%%%%%%%%%
\begin{frame}[fragile]
  \frametitle{Putting the last change to action!}
  This illustrates a key feature of \texttt{Snakemake}. Snakefiles are just Python code. We can make our list into a variable to demonstrate this. Let’s create the variable \texttt{DATS} and use it in our \texttt{zipf\_test} and dats rules.
  \task{Test the following changes!}

  \begin{lstlisting}[language=Python,style=Python]
DATS=['abyss.dat', 'last.dat', 'isles.dat']

rule zipf_test:
    input:
        books=DATS
    output: 'results.txt'
    shell:  'python scripts/zipf_test.py'
            ' {input.books} > {output}'

rule dats:
    input: DATS
  \end{lstlisting}
\end{frame}

%%%%%%%%%%%%%%%%%%%%%%%%%%%%%%%%%%%%%%%%%%%%%%%%%%%%%%%%%%%%%%%%%%%%%%%%%%%%%%%%
\subsection{Supported Functions}

%%%%%%%%%%%%%%%%%%%%%%%%%%%%%%%%%%%%%%%%%%%%%%%%%%%%%%%%%%%%%%%%%%%%%%%%%%%%%%%%
\begin{frame}[fragile]
  \frametitle{Using functions in Snakefiles}
  In our example here, we only have 4 books. But what if we had 700 books to be processed? \newline
  Luckily, \texttt{Snakemake} and Python provide helper functions!
  \task{Start Python in your virtual env and follow along!}
  \begin{lstlisting}[language=Python,style=Python]
from snakemake.io import expand, glob_wildcards
  \end{lstlisting}
  The first function we’ll use is \texttt{expand()}. \texttt{expand()} is used frequently, to expand a snakemake wildcard(s) into a set of filenames:
  \begin{lstlisting}[language=Python,style=Python]
>>> expand('folder/{wildcard1}_{wildcard2}.txt',
...        wildcard1=['a', 'b', 'c'],
...        wildcard2=[1, 2, 3])
  \end{lstlisting}
\end{frame}

%%%%%%%%%%%%%%%%%%%%%%%%%%%%%%%%%%%%%%%%%%%%%%%%%%%%%%%%%%%%%%%%%%%%%%%%%%%%%%%%
\begin{frame}[fragile]
  \frametitle{Using functions in Snakefiles II}
  \question{What is the result?}
  \pause
  \hint[Answer:]{Every permutation of \texttt{wildcard1} and \texttt{wildcard2}!}
  \pause
  Let us try \texttt{glob\_wildcards}, now:
  \begin{lstlisting}[language=Python,style=Python]
glob_wildcards('../books/{example}.txt')
  \end{lstlisting}
  \pause
  Wow! So, easy!
\end{frame}

%%%%%%%%%%%%%%%%%%%%%%%%%%%%%%%%%%%%%%%%%%%%%%%%%%%%%%%%%%%%%%%%%%%%%%%%%%%%%%%%
\subsection{Python Code as Actions}

%%%%%%%%%%%%%%%%%%%%%%%%%%%%%%%%%%%%%%%%%%%%%%%%%%%%%%%%%%%%%%%%%%%%%%%%%%%%%%%%
\begin{frame}[fragile]
  \frametitle{Meeting the \texttt{run} Action}
  \vspace{-0.5em}
  In order to run Python code in \texttt{Snakemake}, we have to replace the \texttt{shell:} action by a \texttt{run:} action:\vspace{-0.5em}
  \task{Add the following to our snakefile:}\vspace{-0.5em}
  \begin{lstlisting}[language=Python,style=Python]
# at the top of the file
import glob

# add this wherever
rule print_book_names:
    run:
        print('These are all the book names:')
        for book in glob.glob('books/*.txt'):
            print(book)

  \end{lstlisting}\vspace{-0.5em}
  \begin{columns}
   \begin{column}{0.2\textwidth}
    And execute:
   \end{column}
   \begin{column}{0.8\textwidth}
\begin{lstlisting}[language=Python,style=Python]
snakemake --cores=1 print_book_names
  \end{lstlisting}
   \end{column}
  \end{columns}

\end{frame}

