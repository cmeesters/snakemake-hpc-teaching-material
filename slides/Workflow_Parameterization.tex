%%%%%%%%%%%%%%%%%%%%%%%%%%%%%%%%%%%%%%%%%%%%%%%%%%%%%%%%%%%%%%%%%%%%%%%%%%%%%%%%
\section{Parametizing your Workflow}

%%%%%%%%%%%%%%%%%%%%%%%%%%%%%%%%%%%%%%%%%%%%%%%%%%%%%%%%%%%%%%%%%%%%%%%%%%%%%%%%
\begin{frame}
    \frametitle{Outline}
    \begin{columns}[t]
        \begin{column}{.5\textwidth}
            \tableofcontents[sections={1-9},currentsection]
        \end{column}
        \begin{column}{.5\textwidth}
            \tableofcontents[sections={10-18},currentsection]
        \end{column}
    \end{columns}
\end{frame}

%%%%%%%%%%%%%%%%%%%%%%%%%%%%%%%%%%%%%%%%%%%%%%%%%%%%%%%%%%%%%%%%%%%%%%%%%%%%%%%%
\begin{frame}
  \frametitle{What is this about?}
   \question[Questions]{\begin{itemize}
                         \item How do we add execution parameters?
                         \item How do we tune scientific parameters?
                        \end{itemize}
                       }
   \docs[Objectives]{\begin{enumerate} 
                      \item Learn to use parameters relevant for the batch systems.
                      \item Learn how to tune \texttt{Snakemake} on the command line.
                      \item Learn how to tune \texttt{Snakemake} with configuration files.
                     \end{enumerate}}
\end{frame}

%%%%%%%%%%%%%%%%%%%%%%%%%%%%%%%%%%%%%%%%%%%%%%%%%%%%%%%%%%%%%%%%%%%%%%%%%%%%%%%%
\subsection{Command Line vs. Configuration File}

%%%%%%%%%%%%%%%%%%%%%%%%%%%%%%%%%%%%%%%%%%%%%%%%%%%%%%%%%%%%%%%%%%%%%%%%%%%%%%%%
\begin{frame}
  \docs{\texttt{Snakemake} has an extensive command line interface (CLI). \emph{Everything} can be configured on the command line. In addition (almost) everything can be specified in a configuration file.}
  \pause
  \begin{exampleblock}{Which parameter goes where? Some rules of thumb:}
    \begin{columns}[t]
      \begin{column}{0.5\textwidth}
        The CLI:
        \begin{itemize}
         \item frequently changing parameters
         \item short parameters
         \item default parameters
        \end{itemize}
      \end{column}
      \begin{column}{0.5\textwidth}
        The Config File:
        \begin{itemize}
         \item non-volantile parameters specific to your analysis (those which merit mentioning in paper should always go into a file)
         \item long parameters
         \item otherwise workflow specific parameters
        \end{itemize}
      \end{column}
    \end{columns}
  \end{exampleblock}
\end{frame}


%%%%%%%%%%%%%%%%%%%%%%%%%%%%%%%%%%%%%%%%%%%%%%%%%%%%%%%%%%%%%%%%%%%%%%%%%%%%%%%%
\subsection{The Command Line}

%%%%%%%%%%%%%%%%%%%%%%%%%%%%%%%%%%%%%%%%%%%%%%%%%%%%%%%%%%%%%%%%%%%%%%%%%%%%%%%%
\begin{frame}[fragile]
  \frametitle{Executor Selection}
  \texttt{Snakemake} lets you select various executors. Not happy with \mogon? Take another cluster or \lhref{https://snakemake.readthedocs.io/en/stable/executor_tutorial/tutorial.html}{Google Lifescience, Tibanna, Kubernetes, \ldots} \newline
  We may happily select the most prominent HPC batch system, the one running on \mogon, too:
  \begin{lstlisting}[language=Bash, style=Shell]
$ snakemake --slurm
  \end{lstlisting}
  Now, \emph{every} rule will submit its jobs as HPC compute jobs.
  \hint{We will learn how to avoid this, soon-ish.}
\end{frame}

%%%%%%%%%%%%%%%%%%%%%%%%%%%%%%%%%%%%%%%%%%%%%%%%%%%%%%%%%%%%%%%%%%%%%%%%%%%%%%%%
\begin{frame}[fragile]
  \frametitle{Default Resources for \texttt{SLURM}}
  Without specifying our SLURM-account and a (default) partition, submitting batch jobs will fail. \texttt{Snakemake} allows to define so-called default resources (using \altverb{--default-resources}). With them our minimal command line becomes:
  \begin{lstlisting}[language=Bash, style=Shell, breaklines=true]
$ snakemake --slurm \
> --default-resources slurm_account=m2_zdvhpc \
>                     slurm_partition=smp
  \end{lstlisting}
  \hint{Please notice the missing quotation marks! All arguments belong to one parameter.}
  \docs{We wrote \texttt{slurm\_...} to distinguish from other non-SLURM accounts and similar stuff.}
\end{frame}

%%%%%%%%%%%%%%%%%%%%%%%%%%%%%%%%%%%%%%%%%%%%%%%%%%%%%%%%%%%%%%%%%%%%%%%%%%%%%%%%
\begin{frame}[fragile]
  \frametitle{The Beauty of Clusters}
  A HPC cluster:
  \begin{itemize}
   \item offers great resources
   \item may hold jobs pending until resources are available!
  \end{itemize}
  \pause
  \texttt{Snakemake} of a semaphore throtteling resource usage, called \altverb{--jobs/-j}. We can now write \altverb{-j unlimited} in place for \altverb{--cores 1}. Let us try
  \begin{lstlisting}[language=Bash, style=Shell, breaklines=true]
$ snakemake --slurm \
> --default-resources slurm_account=m2_zdvhpc \
>                     slurm_partition=smp
> -j unlimited
  \end{lstlisting}
  together.
  \hint{You might want to invoke the \texttt{clean}-rule, first.}
\end{frame}

%%%%%%%%%%%%%%%%%%%%%%%%%%%%%%%%%%%%%%%%%%%%%%%%%%%%%%%%%%%%%%%%%%%%%%%%%%%%%%%%
\begin{frame}[fragile]
  \frametitle{\texttt{SLURM} supporting Features of \texttt{Snakemake}}
  \texttt{Snakemake} will
  \begin{itemize}[<+->]
   \item hold track of your job status (frequency of checks can be adjusted)
   \item cancle your jobs, when itself is stopped
   \item track resource consumption (generated with \altverb{--report [FILE]})
   \item with \altverb{-j unlimited} we allow for an unlimited jumber of spawned jobs!
  \end{itemize}
  \pause
  \warning{Unlimited number of jobs may yield in I/O contention and too many calls to check the job status. Use with care for both issues there is a remedy!}
\end{frame}

%%%%%%%%%%%%%%%%%%%%%%%%%%%%%%%%%%%%%%%%%%%%%%%%%%%%%%%%%%%%%%%%%%%%%%%%%%%%%%%%
\subsection{The Configuration File}

%%%%%%%%%%%%%%%%%%%%%%%%%%%%%%%%%%%%%%%%%%%%%%%%%%%%%%%%%%%%%%%%%%%%%%%%%%%%%%%%
\begin{frame}[fragile]
  \frametitle{The \texttt{Snakemake} \texttt{resources} Section}
  \texttt{Snakemake} rules provide an additional \altverb{resource} section:
  \begin{lstlisting}[language=Python,style=Python]
rule <name>:
   ...
   resources:
      partition='parallel',
      mem_mb=1800,
      cpus_per_task=4
  \end{lstlisting}
  \hint{Note the \textbf{,}!}
  \pause
  \docs{You \emph{may} define \emph{any} resource keyword within any rule.}
\end{frame}

%%%%%%%%%%%%%%%%%%%%%%%%%%%%%%%%%%%%%%%%%%%%%%%%%%%%%%%%%%%%%%%%%%%%%%%%%%%%%%%%
\begin{frame}
  \frametitle{The Configuration File}
  If observed closely you have seen this file:\newline
            {\tiny \DTsetlength{0.2em}{1em}{0.2em}{0.4pt}{.6pt}
\texttt{\$ tree}
\dirtree{%
.1 /.
.2 config.
.3 samples.yaml .
.2 {other stuff}.
}}
 \pause
 \docs{You can store a yaml file with \emph{your} workflow configuration -- which may be combined with the desinger's configuration.}
\end{frame}

%%%%%%%%%%%%%%%%%%%%%%%%%%%%%%%%%%%%%%%%%%%%%%%%%%%%%%%%%%%%%%%%%%%%%%%%%%%%%%%%
\begin{frame}[fragile]
  \frametitle{Using the Configuration File}
  To point to the configuration file, you can add a flag, e.\,g.:
  \begin{lstlisting}[language=Bash, style=Shell]
$ snakemake ... --configfile ./config/samples.yaml  
  \end{lstlisting}
  Yet, you have seen its contents?
  \begin{lstlisting}[language=Bash, style=Shell]
$ cat ./config/samples.yaml
path: '~/workflows/books'
limit: 10
  \end{lstlisting}
  \question{How do pick up resource parameters in out \texttt{Snakefile}?!}
\end{frame} 

\begin{frame}[fragile]
  \frametitle{\Interlude{Introducing Python's \texttt{os} Library}}
  Python has a library to provide functions to deal with operating systems, including file systems. Check out:
  \begin{lstlisting}[language=Python,style=Python]
>>> import os
>>> path = "~/workflows/books"
>>> os.realpath(path)

  \end{lstlisting}

\end{frame}


%%%%%%%%%%%%%%%%%%%%%%%%%%%%%%%%%%%%%%%%%%%%%%%%%%%%%%%%%%%%%%%%%%%%%%%%%%%%%%%%
\begin{frame}[fragile]
  \frametitle{Using the Configuration File - II}
  Within our \texttt{Snakefile} we can access our configuration as any Python dictionary:
  \begin{lstlisting}[language=Python,style=Python]
input_path = config["path"]
  \end{lstlisting}
  \hint{This, however, is useless: if our \texttt{path} is a relative path, every derived path is relative to our current working directory!}
  \pause
  Lukily, \texttt{Snakefile}s are Python-Code:
  \begin{lstlisting}[language=Python,style=Python]
import os

input_path  = os.path.realpath(config['path'])
output_path = os.path.join('results', 
                           os.path.dirname(input_path))

DATS = glob_wildcards(os.path.join(input_path, 
                      "{book}.txt")).book
  \end{lstlisting}
  
\end{frame}

%%%%%%%%%%%%%%%%%%%%%%%%%%%%%%%%%%%%%%%%%%%%%%%%%%%%%%%%%%%%%%%%%%%%%%%%%%%%%%%%
\begin{frame}[fragile]
  \frametitle{Our Workflow}
  \task{Add the following resource to our workflow.}
  \begin{lstlisting}[language=Python,style=Python]
   resources:
      mem_mb=1800
  \end{lstlisting}
  \question{Why? Why not more?}
  \pause
  Because,
  \begin{itemize}
   \item \texttt{ntasks} and \texttt{cpus\_per\_task} default to 1, which is the case here.
   \item account and partition are the same everywhere and we can use this upon submit time (see next slide)
   \item \texttt{mem\_mb} is the same everywhere, too. But: it is usually a resource to be adapter per rule. So, we try this here, too.
  \end{itemize}
\end{frame}

%%%%%%%%%%%%%%%%%%%%%%%%%%%%%%%%%%%%%%%%%%%%%%%%%%%%%%%%%%%%%%%%%%%%%%%%%%%%%%%%
\begin{frame}[fragile]
  \frametitle{Starting our Workflow - One last Time}
  \begin{lstlisting}[language=Bash,style=Shell]
$ snakemake -j unlimited --use-envmodules --slurm \
  --default-ressources account=hpckurs partition=smp
  \end{lstlisting}
  \question{Which warning(s) do turn up? What is the remedy?}
\end{frame}

%%%%%%%%%%%%%%%%%%%%%%%%%%%%%%%%%%%%%%%%%%%%%%%%%%%%%%%%%%%%%%%%%%%%%%%%%%%%%%%%
\begin{frame}[fragile]
  \frametitle{Configuration Files}
  Workflows, once established, should not be altered. All settings go into configuration files. To indicate a configuration file, run with:
  \begin{lstlisting}[language=Bash,style=Shell]
$ snakemake --configfile=<path>
  \end{lstlisting}\pause
  The configuration file itsels is in YAML format and might look like:
  \begin{lstlisting}[language=Python,style=Python]
INPUT_DIR: "/lustre/project/..."
OUTPUT_DIR: "/lustre/project/..."
# environment modules:
VINALC: "bio/VinaLC/1.3.0-gompi-2021b"
  \end{lstlisting}\pause
    Within the Snakefile we can retrieve this information, as it is represented as Python \texttt{dicts}:
  \begin{columns}
     \begin{column}{0.5\textwidth}
       \begin{lstlisting}[language=Bash,style=Shell,basicstyle=\small]
INPUT_DIR=config["INPUT_DIR"]
OUTPU_DIR=config["OUTPUT_DIR"]
       \end{lstlisting}
     \end{column}
     \begin{column}{0.5\textwidth}
      \begin{lstlisting}[language=Bash,style=Shell,basicstyle=\small] 
rule NAME:HPC_Parameterization.tex
    ...
    envmodules:
       config["VINALC"]    
      \end{lstlisting}

     \end{column}
  \end{columns}

\end{frame}
